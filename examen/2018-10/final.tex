
%--------------------------------------------------------------------
%--------------------------------------------------------------------
% Formato para los talleres del curso de Métodos Computacionales
% Universidad de los Andes
%--------------------------------------------------------------------
%--------------------------------------------------------------------

\documentclass[11pt,letterpaper]{exam}
\usepackage[utf8]{inputenc}
\usepackage[spanish]{babel}
\usepackage{graphicx}
\usepackage{tabularx}
\usepackage[absolute]{textpos} % Para poner una imagen en posiciones arbitrarias
\usepackage{multirow}
\usepackage{float}
\usepackage{hyperref}
\decimalpoint

\begin{document}
\begin{center}
{\Large Métodos Computacionales} \\
Examen Final -- 2018-10 \\
\end{center}

\begin{questions}
\question {{\bf C++.}}
Escriba una funci\'on en C++ que tome
como entrada un entero \verb"a" y retorne un flotante con el valor de \verb"a" dividido entre dos.


\question {{\bf Python.}}
Escriba una funci\'on en Python que tome
como entrada dos n\'umeros \verb"a" y \verb"b" y que 
imprima el mensaje \verb"a es mayor que b" si \verb"a" es mayor que \verb"b".


\question {{\bf Unix.}}
Escriba un comando de consola para mover todos los archivos con extensi\'on
\verb".dat" que existen en un directorio a otro directorio llamado
\verb"Prueba". 
Suponga que ejecuta el
comando dentro del directorio que contiene los archivos y el
directorio \verb"Prueba".


\question {{\bf Git.}}
Escriba la secuencia de comandos para incluir un nuevo 
un nuevo  archivo \verb"datos.dat" a un repositorio local y luego sincronizarlo con github.
Suponga que ya se encuentra dentro del repositorio local y que este ya se encuentra
enlazado con github.


\question {{\bf Makefile.}}
Escriba un makefile que: a) compile el c\'odigo \verb"datos.cpp", b) ejecute 
el archivo resultante de la compilaci\'on y redireccione 
la salida  al archivo \verb"datos.dat", y c) ejecute el script \verb"graficas.py", 
para generar la figura  \verb"grafica1.pdf" a partir de los datos en \verb"datos.dat". 
El makefile debe enlazar correcto la secuencia causal entre los diferentes archivos. 
Por ejemplo, si todos los archivos se encuentran al d\'ia y se borra
el archivo \verb"pdf", y luego se ejecuta \verb"make", entonces
solamente se debe ejecutar el comando que genera el \verb"pdf". 


\question {{\bf Principal Component Analysis.}}
Suponga que usted tiene $N$ mediciones de tres cantidades diferentes $x_1$, $x_2$ y
$x_3$. C\'omo se pueden interpretar los auto-valores de la matriz de
covarianza?


\question {{\bf Ecuaciones Diferenciales Ordinarias.}}
Describa la diferencia principal en la construcción de los diferentes
esquemas expl\'icitos de solucion de ecuaciones diferenciales
ordinarias vistos en clase (Euler, Runge-Kutta, etc).


\question {{\bf Ecuaciones Diferenciales Parciales.}}
Considere la ecuaci\'on diferencial parcial de primer orden
$\partial_t u + c \partial_x u=0$, donde $c$ es una constante
positiva.
Escriba un esquema de diferencias finitas estable para resolver esta
ecuaci\'on diferencial. 
Describa la condici\'on que deben cumplir $\Delta x$ y
$\Delta t$ para que la soluci\'on propuesta sea estable.


\question {{\bf Transformada de Fourier.}}
Suponga que tiene una secuencia de datos $x_0, \ldots, x_{N-1}$
equiespaciados temporalmente por $\Delta t=10^{-3}$ con $N=10^6$.
Suponga adem\'as que los resultados de la transformada discreta de Fourier
est\'a dada por $\hat{x}_0, \ldots, \hat{x}_{N-1}$ (con la
  definici\'on vista en clase).
Describa los pasos que debe hacer para construir una nueva secuencia
de datos que borre la informaci\'on de frecuencias mayores a 10Hz de
la secuencia original.


\question {{\bf Diferenciaci\'on.}}
De un ejemplo gr\'afico en el que el m\'etodo de Newton-Raphson falla.

\question {{\bf Integraci\'on.}}
Describa un m\'etodo para integrar num\'ericamente una funci\'on
$f(x)$ en el intervalo $a<x<b$. 
Haga expl\'icitas las condiciones que debe cumplir
$f(x)$ para que su m\'etodos funcione.


\question {{\bf Monte Carlo.}}
$N$ proyectiles con diferentes velocidades iniciales se lanzan
verticalmente. Se miden las $N$ diferentes alturas m\'aximas que 
alcanzan ${h_1,\ldots h_{N}}$. Todas las mediciones tienen la misma 
incertidumbre $\sigma_h$.
Describa como aplicar\'ia un m\'etodo Monte Carlo (i.e. que
utilice n\'umeros aleatorios) para encontrar densidad de probabilidad
del valor de la gravedad, $g$, dadas las $N$ observaciones.


\end{questions}
\end{document}
