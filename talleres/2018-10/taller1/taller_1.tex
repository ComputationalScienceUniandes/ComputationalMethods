
%--------------------------------------------------------------------
%--------------------------------------------------------------------
% Formato para los talleres del curso de Métodos Computacionales
% Universidad de los Andes
%--------------------------------------------------------------------
%--------------------------------------------------------------------

\documentclass[11pt,letterpaper]{exam}
\usepackage[utf8]{inputenc}
\usepackage[spanish]{babel}
\usepackage{graphicx}
\usepackage{tabularx}
\usepackage[absolute]{textpos} % Para poner una imagen en posiciones arbitrarias
\usepackage{multirow}
\usepackage{float}
\usepackage{hyperref}
\decimalpoint

\begin{document}
\begin{center}
{\Large Métodos Computacionales} \\
Taller 1 --- 2018-10\\

\end{center}

%\begin{textblock*}{40mm}(10mm,20mm)
%  \includegraphics[width=3cm]{logoUniandes.png}
%\end{textblock*}

%\begin{textblock*}{40mm}(164mm,20mm)
%  \includegraphics[width=3cm]{logoUniandes.png}
%\end{textblock*}

\vspace{0.3cm}

\noindent
La solución a este taller debe subirse por SICUA antes de las 5:00PM
del lunes 19 de febrero del 2018. 
Si se entrega la tarea antes del lunes 12 de febrero del 2018 a las
11:59PM los ejercicios se van a calificar con el bono indicado. 
\noindent

\vspace{0.3cm}
(10 puntos) Los archivos del c\'odigo  deben subirse en un
\'unico archivo \verb".zip" con el nombre
\verb"NombreApellido_taller1.zip", por ejemplo si su nombre es Miranda
July deber\'ia subir el zip
\verb"MirandaJuly_taller1.zip" al descomprimir el zip debe crearse la
carpeta \verb"MirandaJuly_taller1" y adentro debe estar el c\'odigo.



\vspace{0.3cm}
\begin{questions}

\question {{\bf Geometr\'ia 1} (20 puntos)} 

Escriba un programa en Python que dados dos puntos en tres
dimensiones, $\vec{p}_1$ y $\vec{p}_2$, encuentre los puntos de
intersecci\'on (si existen) con la esfera centrada en $\vec{c}$ y de
radio $R$.  
La funci\'on se define como  \verb"def interseccion(p1, p2, c, R)",
donde \verb"p1", \verb"p2" y \verb"c" son listas de 3 entradas con los
valores $x$, $y$ y $z$ de los vectores y \verb"R" es un float con el
radio de la esfera.  
La funci\'on debe devolver dos listas de 3 entradas con los valores de
la intersecci\'on o \verb"False" si no hay intersecci\'on.  

El nombre del archivo debe ser \verb"geometria_1.py".

\question {{\bf Geometría 2} (20 (25) puntos)}

Escriba un programa en Python que encuentre el radio máximo de una
esfera centrada en un punto con coordenadas
($x_0$, $y_0$, $z_0$) y contenida en un cubo centrado en (0,0,0) y de
lado $(x_0^4+y_0^4+z_0^4)$, donde $|x_0|>1$, $|y_0|>1$, $|z_0|>1$

La funci\'on se define como  \verb"def radio_maximo(p)", donde
\verb"p" es una lista de 3 entradas con los valores $x_0$, $y_0$,
$z_0$ y debe devolver un float con el valor del radio m\'aximo.

El nombre del archivo debe ser \verb"geometria_2.py".

\question {{\bf La flor de la vida} (30 (35) puntos)}

Escriba un programa en Python que dibuje La Flor de la Vida\footnote{\verb"https://es.wikipedia.org/wiki/Flor\_de\_la\_Vida"}

La funci\'on se define como  \verb"def vida()" donde no hay argumentos
de entrada y ning\'un valor devuelto.

El nombre del archivo debe ser \verb"flor.py".

\question {{\bf Temperaturas promedio globales desde 1800}}

El archivo \url{https://data.giss.nasa.gov/gistemp/tabledata_v3/GLB.Ts.txt}
 contiene los datos del cambio de temperatura promedio global con
respecto a la temperatura "base" para los distintos meses de año. La
temperatura base está calculada entre los años 1951 y 1980 y los datos
corresponden al periodo entre 1800 y 2017. Para obtener los cambios en
grados Celsius debe dividir los datos por 100.

En este ejercicio usted debe escribir un script llamado
\verb"temperaturas.sh" que: 

\begin{enumerate}
\item (5 puntos) Descargue los datos de
\url{https://data.giss.nasa.gov/gistemp/tabledata_v3/GLB.Ts.txt} usando el comando \verb"curl".
\item (10 puntos) Guarde en un archivo llamado \verb"TempDic.txt" los años en
que la temperatura promedio de diciembre ha estado por encima del
pse de diciembre.
\item (10 puntos) Guarde en un archivo llamado \verb"TempPomedios.txt"
  el cambio de la temperatura promedio anual y los años.
\item (5 puntos) Grafique los datos anteriores del numeral 3. 
(Cambio en la temperatura promedio anual vs año).
\end{enumerate}

Para hacer la gráfica del numeral 4 escriba un código en python (de
nombre \verb"grafica.py") que se ejecute desde el script
\verb"temperaturas.sh". 



\end{questions}
\end{document}
