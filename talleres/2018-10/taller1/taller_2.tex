
%--------------------------------------------------------------------
%--------------------------------------------------------------------
% Formato para los talleres del curso de Métodos Computacionales
% Universidad de los Andes
%--------------------------------------------------------------------
%--------------------------------------------------------------------

\documentclass[11pt,letterpaper]{exam}
\usepackage[utf8]{inputenc}
\usepackage[spanish]{babel}
\usepackage{graphicx}
\usepackage{tabularx}
\usepackage[absolute]{textpos} % Para poner una imagen en posiciones arbitrarias
\usepackage{multirow}
\usepackage{float}
\usepackage{hyperref}
\decimalpoint

\begin{document}
\begin{center}
{\Large Métodos Computacionales} \\
Tarea 2 --- 2018-10\\

\end{center}

%\begin{textblock*}{40mm}(10mm,20mm)
%  \includegraphics[width=3cm]{logoUniandes.png}
%\end{textblock*}

%\begin{textblock*}{40mm}(164mm,20mm)
%  \includegraphics[width=3cm]{logoUniandes.png}
%\end{textblock*}


Todos los archivos de esta tarea deben guardarse un zip de llamado \verb"minombre_miapellido_tarea2.zip".

\begin{questions}

\question {{\bf Concentración de CO2 en la atmósfera.}}

El objetivo de este ejercicio es explorar los datos de concentraciones
de CO$_2$ en la atmósfera y mirar cuáles han sido las tendencias desde
1960 estudiar la tasa a la cuál se está incrementando dicha
concentración. 

Para esto debe escribir un script \verb"analiza_CO2.sh" para descargar los datos de
\url{ftp://aftp.cmdl.noaa.gov/products/trends/co2/co2_annmean_mlo.txt}
y correr un script de python \verb"analiza.py" que analice los datos.

El script de python debe: 

\begin{itemize}
\item Graficar en el archivo \verb"CO2.png" la concentración de CO$_2$
  atmosférico en función del tiempo.
\item Graficar en el archivo \verb"derivada1_CO2.png" la tasa de
  cambio de dicha concentración en función del tiempo. 
\item Graficar en el archivo \verb"derivada2_CO2.png" la segunda
  derivada de la concentraci\'on de CO$_2$ en funci\'on del tiempo.
  Esta gr\'afica tambi\'en debe incluir una l\'inea horizontal que
  marca la pendiente media de la gr\'afica anterio de tasa de cambio
  calculada entre 1960 y 2016.
\end{itemize}

\question {{\bf Análisis de datos de temperatura}}

En este ejercicio debe hacer un análisis de datos de las temperauras
promedio mensuales en Nottingham durante 20 años. 

Para esto debe escribir un script \verb"minombre_miapellido_temp.sh"
para descargar los datos de
\url{https://raw.githubusercontent.com/vincentarelbundock/Rdatasets/master/csv/datasets/nottem.csv}
y correr un script de python \verb"temperaturas.py" que analice los datos. 

El script de python debe hacer una gr\'afica \verb"temp_analisis.png"
que 

\begin{itemize}
\item Muestre la evoluci\'on temporal de la temperatura. 
\item Marque los m\'aximos de la temperatura.
\item Marque los intervalos donde la temperatura crece.
\end{itemize}


\question {{\bf Distribuci\'on Maxwelliana}}

Para un gas ideal la probabilidad de que la rapidez de una mol\'ecula
sea $v$ est\'a determinada por una densidad de probabilidad
Maxwelliana  

\begin{equation}
\rho = C\frac{v^2}{\sigma^3}\exp{\left(-\frac{1}{2}\frac{v^2}{\sigma^2}\right)}, 
\end{equation}

donde $\sigma=\sqrt{kT/M}$, con $k$ la constante de Boltzmann, $T$ la
temperatura absoluta, $M$ la masa de la mol\'ecula y $C$ es una
constante adimensional tal que $\int_{0}^{\infty}\rho(v)dv=1$.


Escriba en el archivo \verb"maxwell.py" las funciones necesarias para
calcular num\'ericamente como funci\'on de la temperatura en el rango
$100<T/K<1000$ para gases de Helio, Neon y Argon:  
\begin{itemize}
\item La posici\'on del pico de esta densidad de probabilidad, es
  decir, la velocidad $v_{\rm max}$ para la cual $\frac{d \rho}{dv}|_{v_{\rm max}}=0$.
\item La fracci\'on del n\'umero total de \'atomos que tiene una
  rapidez mayor o igual a $300$ m/s, es decir, el cociente $f_{300} =
  \int_{300}^{\infty}\rho(v)dv/\int_0^{\infty}\rho(v)dv$.
\end{itemize}
Los resultados deben guardarse como dos gr\'aficas: \verb"pico.png" y
\verb"fraccion.png". 



\question {{\bf Funci\'on Gamma}}

Escriba en el archivo \verb"gamma.py" una funci\'on en Python
(\verb"def gamma(z)") que devuelve en una variable \verb"float" el
valor de la funci\'on gamma para cualquier n\'umero real positivo
mayor que uno.  

La funci\'on gamma est\'a definida por la integral

\begin{equation}
\Gamma(z) = \int_{0}^{\infty}x^{z-1}e^{-x}dx.
\end{equation}

Para calcular esta integral con los m\'etodos vistos en clase lo m\'as
f\'acil es reescribirla como la suma de dos integrales $\int_0^{\infty} =
\int_{0}^{1} + \int_{1}^{\infty}$ de tal manera que la primera
integral definida es posible calcularla con la regla de Simpson. La
segunda integral indefinida se puede convertir a una integral definida
haciendo un cambio de variable para resolverla tambi\'en con la regla
de Simpson. 



\end{questions}
\end{document}
