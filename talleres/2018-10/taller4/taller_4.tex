
%--------------------------------------------------------------------
%--------------------------------------------------------------------
% Formato para los talleres del curso de Métodos Computacionales
% Universidad de los Andes
%--------------------------------------------------------------------
%--------------------------------------------------------------------

\documentclass[11pt,letterpaper]{exam}
\usepackage[utf8]{inputenc}
\usepackage[spanish]{babel}
\usepackage{graphicx}
\usepackage{tabularx}
\usepackage[absolute]{textpos} % Para poner una imagen en posiciones arbitrarias
\usepackage{multirow}
\usepackage{float}
\usepackage{hyperref}
\decimalpoint

\begin{document}
\begin{center}
{\Large Métodos Computacionales} \\
Taller 4 --- 2018-10\\

\end{center}


\vspace{0.3cm}

\noindent
La solución a este taller debe subirse por SICUA antes de las 5:00PM
del lunes 16 de abril del 2018. 
Si se entrega la tarea antes del lunes 2 de abril del 2018 a las
11:59PM los ejercicios se van a calificar con el bono indicado. 
\noindent

\vspace{0.3cm}
(10 puntos) Los archivos del c\'odigo  deben estar en un \'unico repositorio 
\verb"NombreApellido_taller4", por ejemplo si su nombre es Emma
Goldman el repositorio debe llamarse \verb"EmmaGoldman_taller4".
Al clonarlo debe crearse la carpeta \verb"EmmaGoldman_taller4"
con tres carpetas: \verb"punto_1", \verb"punto_2" y \verb"punto_3"

En la implementaci\'on principal de los algoritmos solicitados la
copia y reutilizaci\'on de c\'odigo de cualquier fuente de internet
(inclu\'ido el repositorio del curso) deja la nota en cero.  

Todas las respuestas deben ser escritas en C++.

\begin{questions}

\question{{\bf Suave}}

(30 (35) puntos) Escriba un c\'odigo para aplicar un suavizado
gaussiano sobre una imagen de formato \verb"png" de entrada. 

El ejecutable debe poder llamarse como 
\verb"./suave imagen.png n_pixel_kernel", donde \verb"imagen.png" es
el nombre del archivo de entrada y \verb"n_pixel_kernel" es el ancho
de la gaussiana del suavizado medida en pixeles.
El resultado se debe guardar en el archivo
\verb"suave.png" con la imagen ya suavizada.


\question {{\bf Filtro}}

(30 (35) puntos)
En clase trabajamos el filtrado de una se\~nal unidimensional
quit\'andole las frecuencias altas y dejando las frecuencias
bajas. 
Ahora usted va a intentar algo similar con una imagen.
Escriba un programa que haga el filtrado de una imagen de
dos maneras. La primera que deje pasar las frecuencias bajas; la
segunda que deje pasar las frecuencias altas.  
En ambos casos implemente un filtro suave. 

Ver la siguiente
referencia: \url{http://paulbourke.net/miscellaneous/imagefilter/}. 


El ejecutable debe poder llamarse como 
\verb"./filtro imagen.png alto" o \verb"./filtro imagen.png bajo".
donde \verb"imagen.png" es el nombre del archivo de entrada y 
\verb"altas/bajas" marca el tipo de frecuencias que deja pasar.
El resultado se debe guardar en el archivo
\verb"altas.png"  o \verb"bajas.png" con la imagen resultante.


\question{{\bf Fourier no uniforme}}

(30 (35) puntos)
El algoritmo que vimos en clase para la transformada de Fourier
discreta es aplicable cuando los datos est\'an distribu\'idos
uniformemente en el tiempo. 

Ahora usted va a implementar un algoritmo que funciona cuando los
puntos no est\'an distribu\'idos de manera uniforme. 
Para esto va a construir primero el polinomio de Lagrange
correspondiente a los $N$ pares de puntos ${t_i,x(t_i)}$.
Con esto va a construir un nuevo conjunto de $N$ pares de puntos que
si se encuentran uniformemente distribuidos en $t$ para calcular la
transformada de Fourier.

El ejecutable debe poder llamarse como 
\verb"./fourier datos.txt" 
donde \verb"datos.txt" es el nombre del archivo de entrada  con dos
columnas: la primera corresponde al tiempo, la segund a la funci\'on
del tiempo.
El resultado se debe guardar en el archivo
\verb"tranformada.txt" donde se guarda el conjunto de $N$ valores
complejos en tres columnas:  la primera corresponde a la frecuencias,
la segunda a la parte real de la transformada y la tercera a la parte imaginaria.



\end{questions}


Nota: Utilice esta libreria \verb"https://github.com/glennrp/libpng" para leer y escribir las
im\'agenes. 
Para la transformada de Fourier debe escribir su propia implementaci\'on.
En todos los casos el ejecutable debe poder crearse con el comando
\verb"make". 
\end{document}.
